\documentclass[t]{beamer}

\usetheme[compress]{Dresden}
\usefonttheme{professionalfonts}
% Insert the frame number at the bottom line
\expandafter\def\expandafter\insertshorttitle\expandafter{%
  \insertshorttitle\hfill\insertframenumber\,/\,\inserttotalframenumber}
% Remove the navigation symbols (one of the two options)
%\usenavigationsymbolstemplate{}
\setbeamertemplate{navigation symbols}{}

\usepackage{graphicx,color,dashbox,amsmath}
\usepackage{hyperref}% embedding hyperlinks [must be loaded after dropping]
\usepackage{amsmath,amsthm,amssymb,amsfonts,latexsym,mathrsfs,wasysym}
\usepackage{pifont}
\usepackage{marvosym}
\usepackage{subfigure}
\usepackage{epstopdf}
\usepackage{soul,color}
\usepackage{threeparttable}% tables with footnotes
\usepackage{dcolumn}% decimal-aligned tabular math columns
\usepackage{float}

%\usepackage[multiple]{footmisc}

\usepackage{colortbl}

\usepackage{tikz}
\usetikzlibrary{calc}
\usetikzlibrary{shapes.geometric}
\usetikzlibrary{arrows.meta}


%\usepackage{enumitem}
%\newlist{arrowlist}{itemize}{1}
%\setlist[arrowlist]{label=$\Rightarrow$}

\newcolumntype{d}{D{.}{.}{-1}}
\graphicspath{{Figures/}}

\logo{\includegraphics[width=3em]{lapid-logo-45}}
\renewcommand{\baselinestretch}{1.2}

% Moshe commands and colors
\newcommand{\fracds}[2]{\ensuremath{\frac{\displaystyle #1}{\displaystyle #2}}}
\newcommand{\rrule}[1]{\rule[#1]{0pt}{0pt}}
\newcommand{\pdf}[2]{\ensuremath{f_{#1}\!\left(#2\rrule{0.9em}\right)}}
\newcommand{\pdfu}[3]{\ensuremath{f_{#1}^{#3}\!\left(#2\rrule{0.9em}\right)}}
\newcommand{\bx}{\ensuremath{\bar{x}}}
\newcommand{\bz}{\ensuremath{\bar{z}}}
\newcommand{\tx}{\ensuremath{\tilde{x}}}
\newcommand{\tX}{\ensuremath{\tilde{X}}}
\newcommand{\tY}{\ensuremath{\tilde{Y}}}
\newcommand{\tz}{\ensuremath{\tilde{z}}}
\newcommand{\ns}{\normalsize}
\newcommand{\smlp}{\mbox{\small $+$}}
\newcommand{\smlm}{\mbox{\small $-$}}
\newcommand{\ba}{\bar a}
\newcommand{\bb}{\bar b}

\newcommand{\sign}[1]{\mbox{\,sign}(#1)}
\newcommand{\kkmo}{k|k-1}
\newcommand{\kk}{{k|k}}
\newcommand{\kpok}{{k+1|k}}
\newcommand{\rR}{\mathbb{R}}

\makeatletter
\newcommand{\mathleft}{\@fleqntrue\@mathmargin0pt}
\newcommand{\mathcenter}{\@fleqnfalse}
\makeatother


% Gilad New Commands
\newcommand{\sgn}[1]{\operatorname{sgn}\left(#1\right)}
\newcommand{\sat}[1]{\operatorname{sat}\left(#1\right)}
%\newcommand{\rrule}[1]{\rule[#1]{0pt}{0pt}}

%beamer@blendedblue
\definecolor{mgreen}{RGB}{40,160,40}
\definecolor{natigreen}{RGB}{50,205,50}
\definecolor{natiblue}{RGB}{0,191,255}
\definecolor{mg}{rgb}{0,0.4,0}%
\definecolor{mypink2}{RGB}{219, 48, 122}

%%%%%%%%%%%%%%%%%%%%%%%%%%

\title{A Projected Lloyd’s Algorithm for Coverage Control Problems}

\author
{Yoav Palti \\ Supervisor: Associate Professor Daniel Zelazo}

\institute[]
{Faculty of Aerospace Engineering, Technion, Haifa, Israel}

\date[MSc Seminar]
{M.Sc. Seminar \\[1ex]
\footnotesize\em December 10, 2018}

\AtBeginSection[]
{
  \begin{frame}
    \frametitle{Table of Contents}
    \tableofcontents[currentsection]
  \end{frame}
}

%%%%%%%%%%%%%%%%%%%%%%%%%%

\begin{document}

\begingroup
% For the first slide only, remove all the text from the header/footer lines
\renewcommand*\insertshorttitle{}
\renewcommand*\insertshortauthor{}
\renewcommand*\insertshortinstitute{}
\renewcommand*\dohead{\rule{0em}{1.45em}}
\begin{frame}[label=sl1]
  \titlepage
\end{frame}
\endgroup

\begin{frame}
\frametitle{Table of Contents}
\tableofcontents
\end{frame}

%%%%%%%%%%%%%%%%%%%%%%%%%%%%%%%%%%%%%%%%%%
%%%%%%%%%%%%%% INTRODUCTION %%%%%%%%%%%%%%
%%%%%%%%%%%%%%%%%%%%%%%%%%%%%%%%%%%%%%%%%%

\section[Introduction]{Introduction}
\subsection[About Me]{}
\begin{frame}[label=abtme]{About Me}
\begin{itemize}
\item Yoav Palti, B.Sc in Aerospace Engineering, Technion, 2012
\item Aeronautical algorithms engineer, IAF, since 2013
\item M.Sc student since 2014
\end{itemize}
\end{frame}

\subsection[Motivation]{}
\begin{frame}[label=sl1]{Motivation}
\begin{itemize}
\item<1-> Covering an area - (relatively) easy
\item<2-> Covering an area with not sufficient amount of sensors - not so easy
\begin{itemize}
\item \textit{Requires better definition of behaviour}
\end{itemize}
\item<3-> Maintain contact with home base (at least in steady state) - hard
\end{itemize}
\end{frame}

\subsection[Problem Formulation]{}
\begin{frame}[label=sl1]{Problem Formulation}

\end{frame}

\subsection[Literature Review]{}
\begin{frame}[label=sl1]{Problem Formulation}

\end{frame}

%%%%%%%%%%%%%%%%%%%%%%%%%%%%%%%%%%%%%%%%%%
%%%%%%%%%%%%% MATH BACKGROUND %%%%%%%%%%%%
%%%%%%%%%%%%%%%%%%%%%%%%%%%%%%%%%%%%%%%%%%

\section[Mathematical Background]{Mathematical Background}
\subsection[Lyaponov Stability]{}
\begin{frame}[label=sl2]{Lyaponov Stability}

\end{frame}

\subsection[Voronoi Partitioning]{}
\begin{frame}[label=sl2]{Voronoi Partitioning}

\end{frame}

\subsection[Voronoi Partitioning]{}
\begin{frame}[label=sl2]{Central Voronoi Tessellations}

\end{frame}

\subsection[Voronoi Partitioning]{}
\begin{frame}[label=sl2]{Lloyd's Algorithm}

\end{frame}

\subsection[Formation Control]{}
\begin{frame}[label=sl2]{Formation Control}

\end{frame}

\subsection[Projection Operator]{}
\begin{frame}[label=sl2]{Projection Operator}

\end{frame}

%%%%%%%%%%%%%%%%%%%%%%%%%%%%%%%%%%%%%%%%%%
%%%%%%%%%%%%% PROBLEM SOLUTION %%%%%%%%%%%
%%%%%%%%%%%%%%%%%%%%%%%%%%%%%%%%%%%%%%%%%%

\section[Problem Solution]{Problem Solution}
\subsection[Lloyd's Algorithm and Formation Control]{}
\begin{frame}[label=sl3]{Lloyd's Algorithm and Formation Control}

\end{frame}

\subsection[Lloyd's Algorithm and Formation Control]{}
\begin{frame}[label=sl3]{Proof}

\end{frame}

\subsection[Projected Lloyd's Algorithm]{}
\begin{frame}[label=sl3]{Projected Lloyd's Algorithm}

\end{frame}

\subsection[Projected Lloyd's Algorithm]{}
\begin{frame}[label=sl3]{Proof}

\end{frame}

\subsection[Problem Solution Algorithm]{}
\begin{frame}[label=sl3]{Problem Solution Algorithm}

\end{frame}

%%%%%%%%%%%%%%%%%%%%%%%%%%%%%%%%%%%%%%%%%%
%%%%%%%%%%%%% Simulations %%%%%%%%%%%
%%%%%%%%%%%%%%%%%%%%%%%%%%%%%%%%%%%%%%%%%%

\section[Simulations]{Simulations}
\subsection[Some Simulation]{}
\begin{frame}[label=sl3]{Some Simulation}

\end{frame}


\end{document}
